\documentclass[a4paper]{exam}

\usepackage{amsmath,amssymb, amsthm}
\usepackage{geometry}
\usepackage{graphicx}
\usepackage{hyperref}
\usepackage{titling}



% Header and footer.
\pagestyle{headandfoot}
\runningheadrule
\runningfootrule
\runningheader{CS/MATH 113, SPRING 2026}{WC 01: \LaTeX practice and logic puzzles}{\theauthor}
\runningfooter{}{Page \thepage\ of \numpages}{}
\firstpageheader{}{}{}

% \printanswers %Uncomment this line

\title{Weekly Challenge 01: \LaTeX\; practice and logic puzzles}
\author{Blingblong} % <=== replace with your student ID, e.g. xy012345
\date{CS/MATH 113 Discrete Mathematics\\Habib University\\Spring 2026}

\qformat{{\large\bf \thequestion. \thequestiontitle}\hfill}
\boxedpoints

\begin{document}
\maketitle

\begin{questions}
  
\titledquestion{Everythings on the table} Draw the truth table of the implication operator.
\begin{solution}
    % Add your solution here
    \[
\begin{array}{c|c|c}
p & q & p \rightarrow q \\ \hline
T & T & T \\
T & F & F \\
F & T & T \\
F & F & T
\end{array}
\]

\end{solution}


\titledquestion{Math is fun ahh} In your past mathematical courses you have come across various mathematical theorems.
Formally state a mathematical theorem that you like. Also briefly talk about why you chose this theorem and what you like about this theorem.
\begin{solution}
    % Add your solution here
    the pythagoras theorem is my favorite theorem because it displays a connection between algebra and geometry.it is widely used in real life contruction applications and it is relatively easy to understand. 
\end{solution}


\titledquestion{How about them apples?}
  \begin{minipage}{.3\linewidth}
  \centerline{\includegraphics[width=\textwidth]{picard}}
\end{minipage}
\begin{minipage}{.65\linewidth}
  The \href{https://en.wikipedia.org/wiki/Replicator_(Star_Trek)}{replicator} aboard USS Enterprise has developed a fault---synthesized apples have insufficient nutrition but are otherwise identical to regular apples. Doctor \href{https://memory-alpha.fandom.com/wiki/Beverly_Crusher}{Beverly Crusher} is on the case. Scanning a bunch of apples, her \href{https://en.wikipedia.org/wiki/Medical_tricorder}{tricorder} can indicate if the bunch contains any faulty apples, but it cannot identify them.
\end{minipage}

    Dr. Crusher is investigating a bunch of 12 apples out of which, 1 is known to be faulty. 
    Describe how she can identify the faulty apple in no more than 4 tricorder scans. Furthermore, what is the minimum number of scans that Dr. Crusher needs to perform in order to guarantee finding the single faulty apple in a bunch of size $n$? Justify your answer.

    \begin{solution}
    % Add your solution here
    Dr.\ Crusher can split the apples into two groups: group A with 2 apples and
group B with 3 apples. Dr.\ Crusher performs a scan on group A.

If the scan identifies a faulty apple, then the faulty apple must be in group A.
Since only two apples remain, a second scan on one of the two apples will identify
whether that apple is faulty or the other one is.

If group A does not contain a faulty apple, then group B must contain the faulty apple.
Dr.\ Crusher then divides group B into two smaller groups, group C with two apples
and group D with one apple. A second scan is performed on group C.

If the scan identifies a faulty apple in group C, group D is discarded. A third scan
on one of the two apples in group C will identify the faulty apple. If the scanned apple
is not faulty, then the other apple must be faulty.

If group C does not contain a faulty apple, then the single apple in group D must
be the faulty one.

In general, Dr.\ Crusher continues dividing the group of apples into two smaller groups
and discards the group that does not contain a faulty apple. After the first scan, she is
left with $\frac{n}{2}$ apples, after the second scan $\frac{n}{4}$ apples, and so on.
After $x$ scans, the number of remaining apples is
\[
\frac{n}{2^x}.
\]

To guarantee that only one apple remains, we require
\[
\frac{n}{2^x} \le 1.
\]
This implies
\[
2^x \ge n,
\]
and therefore
\[
x \ge \log_2 n.
\]

Since the number of scans must be an integer, the minimum number of scans required
to identify the faulty apple is
\[
\lceil \log_2 n \rceil.
\]

    \end{solution}


      
\end{questions}
\end{document}

%%% Local Variables:
%%% mode: latex
%%% TeX-master: t
%%% End:
